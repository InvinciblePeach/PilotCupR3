\documentclass[a4paper, 12pt]{ctexart}
\usepackage{amsmath}
\usepackage{geometry}
\geometry{a4paper,scale=0.8}


\title{机长杯 R3 Sol}
\author{MKurisuqwq}

\begin{document}

\maketitle

\newpage

\section{Contributors}

\begin{table}
    \begin{center}
        \begin{tabular}{l|l|llll}
        id & problem & idea         & std          & data         & check            \\
        \hline
        T0 & 树       & Neat\_Isaac  & Neat\_Isaac  & Neat\_Isaac  & MKurisuqwq       \\
        T1 & 构造      & snowycat1234 & snowycat1234 & snowycat1234 & james1badcreeper \\
        T2 & 宝藏      & snowycat1234 & MKurisuqwq   & MKurisuqwq   & james1badcreeper \\
        T3 & 叉奥      & snowycat1234 & snowycat1234 & snowycat1234 & james1badcreeper \\
        T4 & 原神      & MKurisuqwq   & MKurisuqwq   & MKurisuqwq   & james1badcreeper 
        \end{tabular}
    \end{center}
\end{table}

\section{树}

考察了对于 dfs 序及线段树的掌握. 考虑不换根怎么做, 不难发现子树的 dfs 序是连续的, 因此可以考虑用线段树维护 $dfs$ 序, 支持单点修改, 区间查询.

如果换根呢? 假设我们先按以 $1$ 为根遍历一遍, 如果当前的查询根节点在 $u$ 的子树外, 则 $u$ 当前的子树与以 $1$ 为根时的子树是同一个;
如果当前的查询根节点在 $u$ 的子树内, 则相当于全集去掉 $u$ 的子树里目前根节点所在的那棵子树. 这样我们就把一个询问拆成了序列上至多两个区间的询问, 复杂度 $O(n \log n)$.

\section{构造}

不会平方差公式的举个爪.

不难发现如果 $n$ 能拆分成两个平方数的差时答案 ($0$) 最优, 假定 $n$ 一定能被拆分, 得到:

$$
a^2 - b^2 = n = (a - b)(a + b)
$$

注意到 $a - b \equiv a + b \pmod 2$, 则满足如上拆分的 $n$ 要么是奇数, 要么是 $4$ 的倍数.
不满足如上条件的情况也很好构造, 构造 $p = q = s = n, t = n - 1$ 即可, 既然取不到 $0$, 则 $1$ 一定是最优解.

\section{宝藏}

\subsection{算法一}

设 $n = N + M$.

进行一些观察, 不难得到 $\sum_{S \subseteq U \wedge |S| = K}\sum_{a\in S }p_a=1$ 每个 $p_i$ 贡献了 $\binom{n - 1}{k - 1}$ 次, 则:

\begin{equation}
    \sum p_i = \frac{1}{\binom{n - 1}{k - 1}} \label{con:sump}
\end{equation}

计算每一个宝藏被选到的概率, 他们的和即为答案.

\begin{align}
    P(i) &= \sum_{i \in S} P(S) \\
    &= \sum_{i \in S} \sum_{a \in S} p_a
\end{align}

分别对 $a = i, a \neq i$ 分情况讨论得:

\begin{align}
    = \binom{n - 1}{k - 1}p_i + \sum_{a \neq i} p_a \binom{n - 2}{k - 2}
\end{align}

借助式 \eqref{con:sump} 得到:

\begin{align}
    &= \binom{n - 1}{k - 1}p_i + \binom{n - 2}{k - 2}\left(\frac{1}{\binom{n - 1}{k - 1}} - p_i\right)\\
    &= \frac{\binom{n - 2}{k - 1}}{\binom{n - 1}{k - 1}} + p_i\left[\binom{n - 1}{k - 1} - \binom{n - 2}{k - 2}\right]\\
    &= \frac{k - 1}{n - 1} + p_i \binom{n - 2}{k - 1}
\end{align}

对以上求和 (借助 $i \in N$ 表示 $i$ 是宝藏):

\begin{align}
    E(a) &= \sum_{i \in N} E(i)\\
    &= \frac{m(k - 1)}{n - 1} + \binom{n - 2}{k - 1}\sum_{i \in N} p_i
\end{align}

方差的计算有通过协方差计算的方法, 过于繁杂, 这里不做介绍.

\subsection{算法二}

by james1badcreeper.

可以得知这个概率其实等于先加权随机取一个, 再均匀随机 $K - 1$ 的概率.

直接考虑 dp, 设 $f_{i,j}, g_{i, j}$ 表示考虑前 $i$ 个垃圾/宝藏, 选择 $j$ 个垃圾/宝藏, 的概率之和.
不难得到转移方程:
$$
f_{i, j} = f_{i - 1, j} + \binom{N - 1}{j - 1}\sum p_i
$$

那么我们枚举取到 $A$ 个宝藏, 当前的概率为:
$$
\binom{M}{A} f_{N, k - A} + \binom{N}{k - A} g_{M, A}
$$

乘上 $A$ 就是期望, 乘上 $A^2$ 就是平方的期望, 做差即得方差.

\section{叉奥}

$O(qn^2)$ 做法显然.

一种优化的手段是将 $O(n ^ 2)$ 个区间看作平面上的直线 $Ax + By = C$, 每个询问即计数过该点的直线数量, 将询问离线下来, 用凸包维护所有直线, 在凸包上二分即可.
复杂度 $O(n^2 \log q)$.

原式实际可以转化成:

$$
2(a_l\operatorname{or}a_r)+\bigoplus_{i=l}^ra_i=x(\sum_{i=l}^{r-1}a_i\operatorname{and}a_{i+1})+y(\sum_{i=l}^{r-1}a_i\operatorname{or}a_{i+1})
$$

假定我们固定左端点, 不难发现左边一定小于 $2^{\log_2 a_i + 2}$, 则我们其实只需要考虑等式右边小于这个值的区间即可, 而且等式右边是单调增的, 我们可以考虑向右扩展到不合法的区间就停止.

结论: 这样做的区间个数是 $O(n \log V)$ 的.

证明: 正难则反, 考虑每个点会被他左边的点算几次. 设当前点为 $a_r$, 则对 $l$ 有贡献当且仅当 $a_l \ge \sum_{[l, r]}a_i$, 那么每一个新的合法的 $a_l$ 加进来区间和都会扩大至少一倍, 这种操作最多只有 $O(\log V)$ 次, 则最多只有 $O(\log V)$ 个区间以某个点作右端点, 总共只有 $O(n \log V)$ 个区间.

采取精细化的实现可以做到 $O(n \log V)$ 的维护凸包, 复杂度 $O(n \log V \log q)$.

\section{原神}

\end{document}